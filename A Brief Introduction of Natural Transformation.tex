%Please compile this tex in "pdfLaTeX", and check the package before you compile it!
%It's too hard to draw the commutative diagrams QAQ

\documentclass{article}
\usepackage{amsthm,amsmath,amssymb,mathrsfs,tikz-cd,geometry,hyperref,enumerate}
	\geometry{a4paper,left=1.5cm,right=1.5cm,top=1.0cm,bottom=1.5cm}
	\title{A Brief Introduction of Natural Transformation}
	\author{Fungus}
	\hypersetup{colorlinks=true,linkcolor=blue}

\begin{document}
\maketitle
\theoremstyle{definition}
\newtheorem{defi}{Definition}
\newtheorem{thm}{Theorem}
\newtheorem{lmm}{Lemma}
\newcommand\Ob{\mathrm{Ob}}
\newcommand\Mor{\mathrm{Mor}}
\newcommand\Hom{\mathrm{Hom}}
\newcommand\id{\mathrm{id}}
\newcommand\tid{\mathbf{id}}
\newcommand\C{\mathcal{C}}
\newcommand\D{\mathcal{D}}
\newcommand\1{\mathbf{1}}
\newcommand\iv{^{-1}}
\newcommand\equ{\mathop{\Rightarrow}\limits^{\sim}}

\begin{defi}\label{category}
	A \emph{category} $\C$ consists of:
	\begin{itemize}
		\item A collection $\Ob(\C)$ of \emph{objects} $X,Y,Z\cdots$
		\item A collection 
			$$\Mor(\C)=\bigcup_{X\in\Ob(\C)}{\mathrm{dom}_\C(X)}=\bigcup_{Y\in\Ob(\C)}{\mathrm{cod}_\C(Y)}=\bigcup_{X,Y\in\Ob(\C)}{\Hom_\C(X,Y)}$$
			of \emph{morphisms} $f,g,h\cdots$ with a binary operation ``$\ \circ$'' which is defined on the subclass of $\Mor(\C)\times\Mor(\C)$, where $\mathrm{dom}_\C(X)$ is the \emph{domain} of its elements as well as $\mathrm{cod}_\C(Y)$ is the \emph{codomain} of its elements, and $\Hom_\C(X,Y)$ is the intersection of both. For $W,X,Y,Z\in\Ob(\C)$ and $f\in\Hom_\C(X,Y)\wedge g\in\Hom_\C(Y,Z)$, the binary operation that defines \emph{composite morphism} satisfies:
			\begin{enumerate}
				\item $g\circ f\in\Hom_\C(X,Z)$;
				\item $\forall h\in\Hom_\C(Z,W)[h\circ(g\circ f)=(h\circ g)\circ f]$;
				\item $\forall h\in\Hom_\C(Y,X)\exists \1_X\in\Hom_\C(X,X)[f\circ\1_X=f\wedge\1_X\circ h=h]$.
			\end{enumerate}
	\end{itemize}
	
	It's easy to verify that the \emph{identity morphism} $\1_X$ is unique for all $X\in\Ob(\C)$.\\
	In commutative diagram,
	\[\begin{tikzcd}
			X\arrow[r,"f"] & Y
		\end{tikzcd}\quad\text{means}\quad f\in\Hom_\C(X,Y),\quad
		\begin{tikzcd}
			X\arrow[dd,"f"]\arrow[rrdd,"h"] & & \\ & & \\
			Y\arrow[rr,"g"] & & Z
		\end{tikzcd}\quad\text{means}\quad g\circ f=h.\]
\end{defi}


\begin{defi}\label{functor}
	A \emph{functor} $F:\C\to\D$, between category $\C$ and $\D$, consists the following data:
	\begin{itemize}
		\item Mapping $F:\Ob(\C)\to\Ob(\D)$.
		\item Mapping $F:\Mor(\C)\to\Mor(\D)$, which satisfies:
		\begin{enumerate}
			\item $F[\Hom_\C(X,Y)]\subseteq\Hom_\D(F(X),G(Y))$ for all $X,Y\in\Ob(\C)$;
			\item For all $f,g\in\Mor(\C)$, if $g\circ f$ is defined in $\Mor(\C)$, then $F(g\circ f)=F(g)\circ F(f)$;
			\item $F(\1_X)=\1_{F(X)}$ for all $X\in\Ob(\C)$.
		\end{enumerate}
	\end{itemize}
	In commutative diagram,
	\[\begin{tikzcd}
			\C\arrow[r,"F"] & \D
	\end{tikzcd}\]
	means that $F$ is the functor between $\C$ and $\D$.
	
	For functors $F:\C_1\to\C_2,G:\C_2\to\C_3$, the composition $GF:\C_1\to\C_3$ between both satisfies:
		\[GF(X)=G(F(X))\ \text{and}\ GF(f)=G(F(f))\ \text{for all}\ X\in\Ob(\C_1),f\in\Mor(\C_1).\]
	It's trivial to verify that the composition is also a functor and it satisfy associative law.
		
	For any category $\C$, there exists a \emph{identity functor} $\id_\C:\C\to\C$ that satisfies
		\[\id_\C(X)=X\ \text{and}\ \id_\C(f)=f\ \text{for all}\ X\in\Ob(\C_1),f\in\Mor(\C_1).\]
	It's easy to verify that $F\id_\C=F$ and $\id_\C G=G$ for all functors $F:\C\to\D$ and $G:\D\to\C$.
\end{defi}


\begin{defi}\label{natural transformation}
	The \emph{natural transformation} $\theta$ between functors $F,G:\C\to\D$ is a mapping from $\Ob(\C)$ to $\Mor(\D)$ whose each value satisfies
	\[\theta_X:=\theta(X)\in\Hom_\D(F(X),G(X))\]
	and the commutative diagram below:
	\begin{equation}\begin{tikzcd}\label{ntr}
		F(X) \arrow[r,"\theta_X"] \arrow[d,"F(f)"'] & G(X) \arrow[d,"G(f)"]\\
		F(Y) \arrow[r,"\theta_Y"'] & G(Y),
	\end{tikzcd}\end{equation}
	where $X,Y\in\Ob(\C)$ and $f\in\Hom_\C(X,Y)$. In other words, we can record the above natural transformation as $\theta:F\Rightarrow G$, or in such a commutative diagram:
	\[\begin{tikzcd}
			\C \arrow[r, "F", bend left=50, ""' name=F] \arrow[r, "G"', bend right=50, "" name=G] & \arrow[Rightarrow, from=F, to=G, "\theta"] \D.
	\end{tikzcd}\]
	We may use symbol ``$F(\theta)_X$'' instead of ``$F((\theta)_X)$'' in some particular case (such as there are more than one symbols of natural transformations in the brackets).
	
	For any functor $F:\C\to\D$, there exists a \emph{identity transformation} $\tid^F:F\to F$ that satisfies
	 \[\forall X\in\Ob(\C)[\tid^F_X=\1_{F(X)}].\]
	
	For functors $F:\C_1\to\C_2,G:\C_2\to\C_3$, it's trivial to prove the following formula using the definition of identity morphism and transformation:
	\begin{equation}\label{id}
		\forall X\in\Ob(\C_1)[G(\tid^F_X)=\tid^G_{F(X)}].
	\end{equation}
\end{defi}


\begin{defi}\label{vertical composition}
	For functors $F,G,H:\C\to\D$, natural transformations $\theta:F\Rightarrow G$ and $\psi:G\Rightarrow H$, the element of \emph{vertical composition} of the natural transformations is defined as $(\psi\odot\theta)_X=\psi_X\circ\theta_X$. In commutative diagrams forms,
	\[\begin{tikzcd}
			& \arrow[d,Rightarrow,"\theta"] & \\
			\C \arrow[rr, "F", bend left=75] \arrow[rr, "H"', bend right=75] \arrow[rr, "G" description] &
			\textcolor{white}{s} \arrow[d,Rightarrow,"\psi"] &
			\D\\ & \quad &
		\end{tikzcd}\quad\text{means}\quad\begin{tikzcd}
			\C
			\arrow[bend left=50, rr, "F", ""' name=F]
			\arrow[bend right=50, rr, "H"', "" name=H] & & 
			\arrow[Rightarrow, from=F, to=H, "\psi\odot\theta"] \D.
	\end{tikzcd}\]
	Actually, we need to prove that the definition is well-defined, i.e., to verify that $(\psi\odot\theta)_X\in\Hom_\D(F(X),H(X))$ for all $X\in\Ob(\C)$, it's easy to do so.
\end{defi}


\begin{defi}\label{horizontal composition}
	For functors $F,F':\C_1\to\C_2$ and $G,G':\C_2\to\C_3$, natural transformations $\theta:F\Rightarrow F'$ and $\psi:G\Rightarrow G'$, the element of \emph{horizontal composition} of natural transformations $(\psi\ominus\theta)_X$ is defined as $G'(\theta_X)\circ\psi_{F(X)}=\psi_{F'(X)}\circ G(\theta_X)$. In commutative diagrams forms,
	\[\begin{tikzcd}
			\C_1 \arrow[bend left=50, r, "F", ""' name=F] \arrow[bend right=50, r, "F'"', "" name=F'] &
			\C_2 \arrow[bend left=50, r, "G", ""' name=G] \arrow[bend right=50, r, "G'"', "" name=G'] &
			\arrow[Rightarrow, from=F, to=F', "\theta"] \arrow[Rightarrow, from=G, to=G', "\psi"]\C_3
		\end{tikzcd}\quad{means}\quad\begin{tikzcd}
			\C_1
			\arrow[bend left=50, rr, "GF", ""' name=A]
			\arrow[bend right=50, rr, "G'F'"', "" name=B] & & 
			\arrow[Rightarrow, from=A, to=B, "\psi\ominus\theta"]\C_3,
	\end{tikzcd}\]
	which satisfy
	\begin{equation}\begin{tikzcd}\label{hor}
		GF(X) \arrow[r,"G(\theta_X)"] \arrow[d,"\psi_{F(X)}"'] & GF'(X) \arrow[d,"\psi_{F'(X)}"]\\
		G'F(Y) \arrow[r,"G'(\theta_X)"'] & G'F'(Y).
	\end{tikzcd}\end{equation}
	Actually, we need to prove that the definition is well-defined, i.e., to verify the commutative diagram and that $(\psi\ominus\theta)_X\in\Hom_{\C_3}(GF(X),G'F'(X))$ for all $X\in\Ob(\C_1)$, it's easy to do so observing \hyperref[ntr]{commutative diagram (1)}.
\end{defi}


\begin{thm}
	The vertical and horizontal compositions of natural transformations are natural transformations.	
\end{thm}

\begin{proof}
	For
	\[\begin{tikzcd}
		\C_1
		\arrow[bend left=75, rr, "F", ""' name=F] \arrow[rr, "G" description, "" name=G, ""' name=GG] \arrow[bend right=75, rr, "H"', "" name=H] & &
		\arrow[Rightarrow, from=F, to=G, "\theta"] \arrow[Rightarrow, from=GG, to=H, "\psi"]
		\C_2 \arrow[bend left=75, rr, "F'", ""' name=F'] \arrow[rr, "G'" description, "" name=G', ""' name=GG'] \arrow[bend right=75, rr, "H'"', "" name=H'] & &
		\arrow[Rightarrow, from=F', to=G', "\theta'"] \arrow[Rightarrow, from=GG', to=H', "\psi'"] \C_3
	\end{tikzcd},\]\\
	we need to verify the following commutative diagrams:
		\[(a)\begin{tikzcd}
				F(X) \arrow[dd,"F(f)"'] \arrow[rr,"(\psi\odot\theta)_X"] && H(X) \arrow[dd,"H(f)"]\\ && \\
				F(y) \arrow[rr,"(\psi\odot\theta)_X"'] && H(y)
			\end{tikzcd}\quad\text{and (b)}\quad\begin{tikzcd}
				GF(X) \arrow[dd,"GF(f)"] \arrow[rr,"(\theta'\ominus\theta)_X"] && G'F'(X) \arrow[dd,"G'F'(f)"]\\ && \\
				GF(Y) \arrow[rr,"(\theta'\ominus\theta)_Y"'] && G'F'(Y)
		\end{tikzcd}.\]
		
	From (a), we have
	\begin{align}
		& H(f)\circ(\psi\odot\theta)_X\tag{Assumption}\\
		=& H(f)\circ\psi_X\circ\theta_X\tag{Def: vertical composition}\\
		=& (\psi_Y\circ G(f))\circ\theta_X\tag{Property of natural transformation $\psi$}\\
		=& \psi_Y\circ\theta_Y\circ F(f)\tag{Property of natural transformation $\theta$}\\
		=& (\psi\odot\theta)_X\circ F(f),\tag{Def: vertical composition}
	\end{align}
	thus $(\psi\odot\theta)$ is natural transformation.\\
		
	From (b), we have
	\begin{align}
		& G'F'(f)\circ(\theta'\ominus\theta)_X\tag{Assumption}\\
		=& G'F'(f)\circ\theta'_{F'(X)}\circ G(\theta_X)\tag{Def: horizontal composition}\\
		=& \theta'_{F'(Y)}\circ GF'(f)\circ G(\theta_X)\tag{Property of natural transformation $\theta'$}\\
		=& \theta'_{F'(Y)}\circ G(F'(f)\circ G(\theta_X))\tag{Property of functor $G$}\\
		=& \theta'_{F'(Y)}\circ G(\theta_Y\circ F(f))\tag{Property of natural transformation $\theta$}\\
		=& \theta'_{F'(Y)}\circ G(\theta_Y)\circ GF(f)\tag{Property of functor $G$}\\
		=& (\theta'\ominus\theta)_Y\circ GF(f)\tag{Def: horizontal composition},
	\end{align}
	thus $(\psi\ominus\theta)$ is natural transformation.\\
		
	{\bf What's more}, we can prove that $(\psi\odot\theta)\ominus(\psi'\odot\theta')=(\psi'\ominus\psi)\odot(\theta'\ominus\theta)$:
	\begin{align}
		 & ((\psi'\odot\theta')\ominus(\psi\odot\theta))_X\tag{Assumption}\\
		=& (\psi'\odot\theta')_{H(X)}\circ F'(\psi\odot\theta)_X\tag{Def: horizontal composition}\\
		=& \psi'_{H(X)}\circ\theta_{H(X)}\circ F'(\theta_X)\circ F'(\theta_X)\tag{Def: vertical composition, Property of functor $F'$}\\
		=& \psi'_{H(X)}\circ(G'(\psi_X)\circ \theta'_{G(X)})\circ F'(\theta_X)\tag{\hyperref[hor]{Commutative diagram (3)}}\\
		=& (\psi'\ominus\psi)_X\circ(\theta'\ominus\theta)_X\tag{Def: horizontal composition}\\
		=& ((\psi'\ominus\psi)\odot(\theta'\ominus\theta))_X,\tag{Def: vertical composition}
	\end{align}
	where $X\in\Ob(\C)$.
\end{proof}

\begin{thm}
	Both vertical and horizontal compositions of natural transformations satisfy associative law.
\end{thm}

\begin{proof}
	For vertical composition, observe the following commutative diagram and natural transformations:

	For
	\[\begin{tikzcd}
		\C
		\arrow[r, bend left =90, "F"]
		\arrow[r, bend left =35, "G" description] 
		\arrow[r, bend right=35, "H" description]
		\arrow[r, bend right=90, "K"'] &
		\D
	\end{tikzcd}\quad\text{as well as}\quad
	\left\{\begin{aligned}
		& \theta:F\Rightarrow G,\\
		& \psi:G\Rightarrow H,\\
		& \phi:H\Rightarrow K,
	\end{aligned}\right.\]
	it's trivial to prove that $((\phi\odot\psi)\odot\theta)_X=(\phi\odot(\psi\odot\theta))_X$ for all $X\in \Ob(\C)$, thus the vertical composition satisfies associative law.

	For
	\[\begin{tikzcd}
		\C_1 \arrow[bend left=50, r, "F", ""' name=F] \arrow[bend right=50, r, "F'"', "" name=F'] &
		\C_2 \arrow[bend left=50, r, "G", ""' name=G] \arrow[bend right=50, r, "G'"', "" name=G'] &	
		\C_3 \arrow[bend left=50, r, "H", ""' name=H] \arrow[bend right=50, r, "H'"', "" name=H'] &
		\arrow[Rightarrow, from=F, to=F', "\theta"]
		\arrow[Rightarrow, from=G, to=G', "\psi"]
		\arrow[Rightarrow, from=H, to=H', "\phi"]
		\C_4,
	\end{tikzcd}\]
	we have
	\begin{align}
		 & (\phi\ominus(\psi\ominus\theta))_X\tag{Assumption}\\
		=& \phi_{G'F'(X)}\circ H(\psi\ominus\theta)_X\tag{Def: horizontal composition}\\
		=& \phi_{G'F'(X)}\circ H(\psi_{F'(X)}\circ G(\theta_X))\tag{Ditto}\\
		=& \phi_{G'F'(X)}\circ H(\psi_{F'(X)})\circ HG(\theta_X)\tag{Property of functor $H$}\\
		=& (\phi\ominus\psi)_{F'(X)}\circ HG(\theta_X)\tag{Def: horizontal composition}\\
		=& ((\phi\ominus\psi)\ominus\theta)_X\tag{Ditto},
	\end{align}
	thus the horizontal composition satisfies associative law.
\end{proof}


\begin{thm}
	Observe the following natural commutative:
	\[\begin{tikzcd}
		\C_1 \arrow[r,"F"] &
		\C_2 \arrow[r, bend left =50, "G", ""' name=G] \arrow[r, bend right=50, "H"', "" name=H] &
		\C_3 \arrow[r, "K"] &
		\arrow[Rightarrow, from=G, to=H, "\theta"]
		\C_4,
	\end{tikzcd}\]
	we have
	\begin{align}
		\theta\odot\tid^G=\theta=&\tid^H\odot\theta,\label{1}\\
		(\theta\ominus\tid^F)_X=&\theta_{F(X)},\label{2}\\
		(\tid^K\ominus\theta)_Y=&K(\theta_Y),\label{3}
	\end{align}
	for all $X\in\Ob(\C_1),Y\in\Ob(\C_2)$.
\end{thm}

\begin{proof}
	It's trivial to prove \hyperref[1]{(4)} and using definition of identity transformation and vertical composition.
	
	For \hyperref[2]{(5)}, we have
	\begin{align}
		 & (\theta\ominus\tid^F)_X\tag{Assumption}\\
		=& \theta_{F(X)}\circ G(\tid^F_X)\tag{Def: horizontal composition}\\
		=& \theta_{F(X)}\circ \tid^G_{F(X)}\tag{\hyperref[id]{formula 2}}\\
		=& \theta_{F(X)}\tag{\hyperref[1]{formula 4}}
	\end{align}
	for all $X\in\Ob(\C_1)$.
	
	For \hyperref[3]{(6)}, we have
	\begin{align}
		 & (\tid^K\ominus\theta)_Y\tag{Assumption}\\
		=& K(\theta_Y)\circ\tid^K_{G(Y)}\tag{Def: horizontal composition}\\
		=& K(\theta_Y)\circ K(\tid^G_Y)\tag{\hyperref[id]{formula 2}}\\
		=& K(\theta_Y\circ\tid^G_Y)\tag{Property of functor $K$}\\
		=& K(\theta_Y)\tag{\hyperref[1]{formula 4}}
	\end{align}
	for all $Y\in\Ob(\C_2)$.
\end{proof}


\begin{lmm}[$\mathsf{ZF}$]\label{map}
	For any sets $A,B$ and mapping $f:A\to B$, we have
		\[\exists g:B\to A[g\circ f=\id_A]\iff f\ \text{is a injection}\ \iff\forall C\forall h,h':C\to A[f\circ h=f\circ h'\Longrightarrow h=h'],\]
		\[\exists g:B\to A[f\circ g=\id_B]\Longrightarrow f\ \text{is a surjection}\ \iff\forall C\forall h,h':B\to C[h\circ f=h'\circ f\Longrightarrow h=h'],\]
	and $f$ is a surjection $\Longrightarrow \exists g:B\to A[f\circ g=\id_B]$ can be proved in $\mathsf{ZFC}$.
	It's easy to see that $f$ is bijection if and only if it has both left and right inversal mappings, and obviously the two inversal mappings are the same one, which is unique.
\end{lmm}


\begin{defi}\label{inverse}
	Consider $X,Y\in\Ob(\C)$ and $f\in\Hom_\C(X,Y)$. If there exists a morphism $g\in\Hom_\C(Y,X)$ that $f\circ g=\1_Y$, then we say $g$ is the \emph{right inverse} of it; if $g\circ f=\1_X$, then we say $g$ is the \emph{left inverse} of it. If $f$ has both inverses, then we say $f$ is \emph{isomorphic}, it's easy to verify that the two inverses  are the same one, which is unique as well, so we say $f\iv:=g$ is the \emph{inverse} (\emph{inversal morphism}) of it.
	
	What's more, it's trivial to verify that the composition of isomorphic morphisms are also isomorphic, so we can find that the collection of \emph{automorphisms} $\mathrm{Aut}_\C(X):=\{f\in\Hom_\C(X,Y)\,|\,f\ \text{is isomorphic}\}$ is a group $\left<\mathrm{Aut}_\C(X),\circ,\1_X\right>$. Therefore if $f$ has a left inverse, then $f$ is a \emph{injective morphism}, which satisfies the \emph{left cancellation law}: $\forall Z\in\Ob(C)\forall g,h\in\Hom_\C(Z,X)[f\circ g=f\circ h\iff g=h]$; if $f$ has a right inverse, then $f$ is a \emph{surjective morphism}, which satisfies the \emph{right cancellation law}: $\forall Z\in\Ob(C)\forall g,h\in\Hom_\C(Y,Z)[g\circ f=h\circ f\iff g=h]$.
	
	Consider a functor $F:\C\to\D$, we can define the inverse of functor in the same way: If there exists a functor $G:\D\to\C$ that $GF=\id_\C$ and $FG=\id_\D$, then we say $F$ is isomorphic ($F$ is a \emph{isomorphism} between $\C$ and $\D$), and $F\iv:=G$ is the unique inverse (\emph{inversal functor}) of $F$. From \hyperref[map]{Lemma 1} we know that $F$ is isomorphic if and only if  $F\upharpoonright_{\Ob(\C)}$ and $F\upharpoonright_{\Mor(\C)}$ are both bijection.
	
	Consider the following diagram:
	\[\begin{tikzcd}
		\C 
		\arrow[rr, bend left =75, "F", ""' {name=U1, near start}, ""' {name=U2, near end}]
		\arrow[rr, bend right=75, "G"', "" {name=D1, near start}, ""  {name=D2, near end}]
		&&\D.
		\arrow[Rightarrow, "\theta", from=U1, to=D1]
		\arrow[Rightarrow, "\psi"  , from=D2, to=U2]
    \end{tikzcd}\]
	If $\psi\odot\theta=\tid^F$ and $\theta\odot\psi=\tid^G$, then we say $\theta$ is isomorphic between $F$ and $G$, and $\theta\iv:=\psi$ is the unique inverse (\emph{inversal transformation}) of $\theta$, we record it as $\theta:F\mathop{\Rightarrow}\limits^{\sim}G$. It's trivial to prove that $\theta$ is isomorphic if and only if $\theta_X$ is isomorphic for each $X\in\Ob(\C)$, thus we have $(\theta\iv)_X=(\theta_X)\iv$, we abbreviate it as ``$\theta\iv_X$''.
	
	For functors $F:\C\to\D$ and $G:\D\to\C$, if there exist isomorphic $\theta:FG\equ\id_{\D}$ and $\psi:GF\equ\id_{\C}$, then we say $G$ is the \emph{quasi-inverse} of $F$, and $F$ is a \emph{equivalence} between $\C$ and $\D$. If there exist an equivalence between two categories $\C$ and $\D$, we say they are \emph{equivalent} and record it as $\C\simeq\D$.
	
	It's tirvial to verify that the composition of any isomorphic morphisms/functors/transformations is isomorphic, and the composition of any equivalences are equivalence.
\end{defi}

\end{document}
