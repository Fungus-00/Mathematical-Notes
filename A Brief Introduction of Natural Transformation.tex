%Please compile this tex in "pdfLaTeX", and check the package before you compile it!
%It's too hard to draw the commutative diagrams QAQ

\documentclass{article}
\usepackage{amsthm,amsmath,amssymb,mathrsfs,tikz-cd,geometry,hyperref,enumerate,xcolor}
	\geometry{a4paper,left=1.5cm,right=1.5cm,top=1.25cm,bottom=1.5cm}
	\title{A Brief Introduction of Natural Transformation}
	\author{Fungus}
	\hypersetup{colorlinks=true,linkcolor=blue}

\begin{document}
\maketitle
\renewcommand{\thefootnote}{\color{red}{*}}
\theoremstyle{definition}
\newtheorem{defi}{Definition}
\newtheorem{thm}{Theorem}
\newtheorem{lmm}{Lemma}
\newtheorem{exm}{Example}
\newtheorem{cor}{Corollary}
\newcommand\Ob{\mathrm{Ob}}
\newcommand\Mor{\mathrm{Mor}}
\newcommand\Hom{\mathrm{Hom}}
\newcommand\id{\mathrm{id}}
\newcommand\op{^\mathrm{op}}
\newcommand\zfc{\mathsf{ZFC}}
\newcommand\con{\mathrm{Con}}
\newcommand\tid{\mathbf{id}}
\newcommand\C{\mathcal{C}}
\newcommand\D{\mathcal{D}}
\newcommand\1{\mathbf{1}}
\newcommand\iv{^{-1}}
\newcommand\equ{\mathop{\Rightarrow}\limits^{\sim}}

\leftline{\it These are my notes of basic category theory.}


\begin{defi}\label{category}
	A \emph{category} $\C$ consists of:
	\begin{itemize}
		\item A collection $\Ob(\C)$ of \emph{objects} $X,Y,Z\cdots$
		\item A collection 
			$$\Mor(\C)=\bigcup_{X\in\Ob(\C)}{\mathrm{dom}_\C(X)}=\bigcup_{Y\in\Ob(\C)}{\mathrm{cod}_\C(Y)}=\bigcup_{X,Y\in\Ob(\C)}{\Hom_\C(X,Y)}$$
			of \emph{morphisms} $f,g,h\cdots$ with a binary operation ``$\circ$'' which is defined on the subclass of $\Mor(\C)\times\Mor(\C)$, where $\mathrm{dom}_\C(X)$ is the \emph{domain} of its elements as well as $\mathrm{cod}_\C(Y)$ is the \emph{codomain} of its elements, and \emph{homomorphism} $\Hom_\C(X,Y)$ is the intersection of both.
				\footnote{A few textbooks may require that the different homomorphisms in the same category are disjoint, however, it will cause a problem in defining functor category, we will explain this in detail after the definition of functor category.}
			For $W,X,Y,Z\in\Ob(\C)$ and $f\in\Hom_\C(X,Y)\wedge g\in\Hom_\C(Y,Z)$, the binary operation that defines \emph{composite morphism} $g\circ^\C f$ (which is abbreviated as $g\circ f$) satisfies:
			\begin{enumerate}
				\item $g\circ f\in\Hom_\C(X,Z)$;
				\item $\forall h\in\Hom_\C(Z,W)[h\circ(g\circ f)=(h\circ g)\circ f]$;
				\item $\forall h\in\Hom_\C(Y,X)\exists \1^\C_X\in\Hom_\C(X,X)[f\circ\1^\C_X=f\wedge\1^\C_X\circ h=h]$.
			\end{enumerate}
	\end{itemize}
	
	It's easy to verify that the \emph{identity morphism} $\1^\C_X$ (which is abbreviated as $\1_X$) is unique for all $X\in\Ob(\C)$.\\
	
	In commutative diagram,
	\[\begin{tikzcd}
			X\arrow[->,>=stealth,r,"f"] & Y
		\end{tikzcd}\quad\text{means}\quad f\in\Hom_\C(X,Y),\quad
		\begin{tikzcd}
			X\arrow[->,>=stealth,dd,"f"]\arrow[->,>=stealth,rrdd,"h"] & & \\ & & \\
			Y\arrow[->,>=stealth,rr,"g"] & & Z
	\end{tikzcd}\quad\text{means}\quad g\circ f=h.\]
\end{defi}


\begin{exm}[$\mathsf{ZF}$]
	Consider a category $\mathsf{Rel}$:
	\begin{itemize}
		\item Objects are all sets.
		\item Homomorphism between any sets $X,Y$ is the power set of binary relations $\mathscr{P}(X\times Y)$.
		\item The composition of morphisms is the composition of binary relations.
		\item Identity morphism $\1_X$ is the identity mapping $\id_X$.
	\end{itemize}
	Obviously, it's indeed a category. This example shows that the morphisms are not only mappings, they may have looser structures. Compared to this, morphisms are more like binary relations.
\end{exm}


\begin{defi}\label{subcategory, opposite category, small category}
	$\C'$ is the \emph{subcategory} of category $\C$ if:
	\begin{itemize}
		\item $\C'$ is a category;
		\item $\Ob(\C')\subseteq\Ob(\C)$;
		\item $\forall X,Y\in\Ob(\C')[\Hom_{\C'}(X,Y)\subseteq\Hom_\C(X,Y)]$;
		\item $\forall f,g\in\Mor(\C')[f\circ^{\C'}g=f\circ^{\C}g]$ (if $f\circ^{\C}g$ is defined);
		\item for all $X\in\Ob(\C')$, the identity morphism $\1_X$ in $\C'$ is also that in $\C$.
	\end{itemize}
	In particular, if $\forall X,Y\in\Ob(\C')[\Hom_{\C'}(X,Y)=\Hom_\C(X,Y)]$, then we say 	$\C'$ is the \emph{full subcategory} of $\C$.
	
	A \emph{opposite} category $\C\op$ of category $\C$ satisfies:
	\begin{itemize}
		\item $\Ob(\C\op)=\Ob(\C)$;
		\item $\forall X,Y\in\Ob(\C\op)[\Hom_{\C\op}(X,Y)=\Hom_{\C}(Y,X)]$;
		\item $\forall f,g\in\Mor(\C\op)[f\circ\op g=g\circ^\C f]$ (if $g\circ^\C f$ is defined);
		\item for all $X\in\Ob(\C\op)$, the identity morphism $\1_X$ in $\C\op$ is also that in $\C$.
	\end{itemize}
	It's easy to verify that $\C$ is also a category, and we have $(\C\op)\op=\C$. $\C\op$ has the symmetric algebraic properties as $\C$.
	
	A category $\C$ is called \emph{small} if both $\Ob(\C)$ and $\Mor(\C)$ are sets in $\zfc$ but not proper class,
	 \footnote{``$X$ is a set in $\zfc$'' has two meanings: we can prove $X$ exists in $\zfc$, i.e., $\zfc\vdash_\mathbf{H}\exists X$; or the existence of $X$ in $\zfc$ is consistent with $\zfc$, i.e., $\vdash_\mathbf{H}\mathrm\con(\zfc)\to\con(\zfc+\exists X)$, where $\mathbf{H}$ means the \emph{Frege-Hilbert first-order logic axiomatic system}. The meaning in the text is the former. Of course, to prove the consistency, we often need to add extra axioms. The provability of $\zfc$ is limited, so we can only define the set in the model $(V_\kappa,\in)$, where $\kappa$ is the least strongly inaccessible cardinal, but that's enough.}
	and \emph{large} otherwise. A \emph{locally small} category is a category such that for all objects $X$ and $Y$, $\Hom(X,Y)$ is a set in $\zfc$, called a \emph{homset}.
\end{defi}


\begin{defi}\label{functor}
	A \emph{functor} $F:\C\to\D$, between category $\C$ and $\D$, consists the following data:
	\begin{itemize}
		\item Mapping $F:\Ob(\C)\to\Ob(\D)$.
		\item Mapping $F:\Mor(\C)\to\Mor(\D)$, which satisfies:
		\begin{enumerate}
			\item $F[\Hom_\C(X,Y)]\subseteq\Hom_\D(F(X),G(Y))$ for all $X,Y\in\Ob(\C)$;
			\item For all $f,g\in\Mor(\C)$, if $g\circ f$ is defined in $\Mor(\C)$, then $F(g\circ f)=F(g)\circ F(f)$;
			\item $F(\1_X)=\1_{F(X)}$ for all $X\in\Ob(\C)$.
		\end{enumerate}
	\end{itemize}
	
	In commutative diagram,
	\[\begin{tikzcd}
			\C\arrow[r,"F"] & \D
	\end{tikzcd}\]
	means that $F$ is the functor between $\C$ and $\D$.
	
	For functors $F:\C_1\to\C_2,G:\C_2\to\C_3$, the composition $GF:\C_1\to\C_3$ between both satisfies:
		\[GF(X)=G(F(X))\ \text{and}\ GF(f)=G(F(f))\ \text{for all}\ X\in\Ob(\C_1),f\in\Mor(\C_1).\]
	It's trivial to verify that the composition is also a functor and it satisfy associative law.
	
	For any category $\C$, there exists a \emph{identity functor} $\id_\C:\C\to\C$ that satisfies
		\[\id_\C(X)=X\ \text{and}\ \id_\C(f)=f\ \text{for all}\ X\in\Ob(\C_1),f\in\Mor(\C_1).\]
	It's easy to verify that $F\id_\C=F$ and $\id_\C G=G$ for all functors $F:\C\to\D$ and $G:\D\to\C$.
\end{defi}


\begin{defi}\label{natural transformation}
	The \emph{natural transformation} $\theta$ between functors $F,G:\C\to\D$ is a mapping from $\Ob(\C)$ to $\Mor(\D)$ whose each value satisfies
	\[\theta_X:=\theta(X)\in\Hom_\D(F(X),G(X))\]
	and the commutative diagram below:
	\begin{equation}\begin{tikzcd}\label{ntr}
		F(X) \arrow[->,>=stealth,r,"\theta_X"] \arrow[->,>=stealth,d,"F(f)"'] & G(X) \arrow[->,>=stealth,d,"G(f)"]\\
		F(Y) \arrow[->,>=stealth,r,"\theta_Y"'] & G(Y),
	\end{tikzcd}\end{equation}
	where $X,Y\in\Ob(\C)$ and $f\in\Hom_\C(X,Y)$. In other words, we can record the above natural transformation as $\theta:F\Rightarrow G$, or in such a commutative diagram:
	\[\begin{tikzcd}
			\C \arrow[r, "F", bend left=50, ""' name=F] \arrow[r, "G"', bend right=50, "" name=G] & \arrow[Rightarrow, from=F, to=G, "\theta"] \D.
	\end{tikzcd}\]
	
	We may use symbol ``$F(\theta)_X$'' instead of ``$F((\theta)_X)$'' in some particular case (such as there are more than one symbols of natural transformations in the brackets).
	
	For any functor $F:\C\to\D$, there exists a \emph{identity transformation} $\tid^F:F\to F$ that satisfies
	 \[\forall X\in\Ob(\C)[\tid^F_X=\1_{F(X)}].\]
\end{defi}


\begin{exm}
	Consider two \emph{finite} categories $\C,\D$ whose $\Ob(\C)=\{X,Y\}$ and $\Ob(\D)=\{\{a,b\},\{1,2\},\{3,4\},\{c,d\}\}$. There are three morphisms in $\C$: $\1_X$, $\1_Y$ and $f\in\Hom_\C(X,Y)$. Consider four functors $F,F',G,G':\C\to\D$ such that
	\[F(X)=\{a,b\}=F'(X),G(Y)=\{c,d\}=G'(Y).\]
	And consider two natural transformations $\theta,\psi:F\Rightarrow G$, and all the morphisms (mappings) in $\D$ except identity morphisms are shown below:
	\[\begin{tikzcd}
	& a \arrow[r,no head,dashed,"F(X)"] \arrow[->,>=stealth,ldd,color=red] \arrow[->,>=stealth,rrdd,color=green] \arrow[->,>=stealth,rrd,color=cyan] \arrow[->,>=stealth,ddd]
	& b \arrow[->,>=stealth,lldd,color=red] \arrow[->,>=stealth,rd,color=cyan] \arrow[->,>=stealth,rdd,color=green] \arrow[->,>=stealth,lddd] & \\
	  1 \arrow[->,>=stealth,rdd,color=yellow] \arrow[->,>=stealth,rrdd,color=blue] && 
	& 3 \arrow[d,no head,dashed] \arrow[->,>=stealth,lldd,color=brown]\\
	  2 \arrow[u,no head,dashed] \arrow[->,>=stealth,rd,color=yellow] \arrow[->,>=stealth,rd,color=blue,bend right] &&
	& 4 \arrow[->,>=stealth,lld,color=brown]\\
	& c 
	& d \arrow[l,no head,dashed,"G(Y)"] &         
	\end{tikzcd}\]
	Where the arrow with different color means different mapping, balck arrows mean the morphism $k$, and the elements connected by one dashed line belong to the same set. There are 7 isomorphisms (except identity morphisms) in $\D$ in total, it's easy to see that $\D$ is indeed a category (we just need to verify that the compositions of any morphisms in $\D$ are also morphisms in it).
	
	\begin{itemize}
		\item Consider the following combination of objects and morphisms:
			\[G(X)=\{1,2\}=G'(X),F(Y)=\{3,4\}=F'(Y);\]
			\centerline{\textcolor{red}{red:}$\theta_X$, \textcolor{yellow}{yellow}:$G(f)$, \textcolor{blue}{blue:}$G'(f)$, \textcolor{cyan}{cyan:}$F(f)$, \textcolor{green}{grenn:}$F'(f)$, \textcolor{brown}{brown:}$\theta_Y$.}\\
		The four functors are indeed functors. What's more, it's trivial to verify that
		\[\theta_Y\circ F'(f)=\theta_Y\circ F(f)=k=G(f)\circ\theta_X=G'(f)\circ\theta_X,\]
		thus we know $\theta$ is indeed a natural transformation, and obviously $\theta$ have more that one ``sources'' and ``targets''.
		
		\item Consider the following combination of objects and morphisms:
			\[G(X)=\{3,4\},F(Y)=\{1,2\};\]
			\centerline{\textcolor{red}{red:}$F(f)$, \textcolor{yellow}{yellow}:$\theta_Y$, \textcolor{blue}{blue:}$\psi_Y$, \textcolor{cyan}{cyan:}$\theta_X$, \textcolor{green}{grenn:}$\psi_X$, \textcolor{brown}{brown:}$G(f)$.}\\
		The two functors are indeed functors. What's more, it's trivial to verify that
		\[\theta_Y\circ F(f)=\psi_Y\circ F(f)=k=G(f)\circ\theta_X=G(f)\circ\psi_X,\]
		thus we know $\theta$ and $\psi$ are indeed natural transformations between $F$ and $G$.
	\end{itemize}
	These examples show us that one natural transformation can rely on different functors, and there may be different natural transformations between two functors. 
\end{exm}


\begin{defi}\label{vertical composition}
	For functors $F,G,H:\C\to\D$, natural transformations $\theta:F\Rightarrow G$ and $\psi:G\Rightarrow H$, the element of \emph{vertical composition} of the natural transformations is defined as $(\psi\odot\theta)_X=\psi_X\circ\theta_X$. In commutative diagrams forms,
	
	\[\begin{tikzcd}
			& \arrow[d,Rightarrow,"\theta"] & \\
			\C \arrow[rr, "F", bend left=75] \arrow[rr, "H"', bend right=75] \arrow[rr, "G" description] &
			\textcolor{white}{s} \arrow[d,Rightarrow,"\psi"] &
			\D\\ & \quad &
		\end{tikzcd}\quad\text{means}\quad\begin{tikzcd}
			\C
			\arrow[bend left=50, rr, "F", ""' name=F]
			\arrow[bend right=50, rr, "H"', "" name=H] & & 
			\arrow[Rightarrow, from=F, to=H, "\psi\odot\theta"] \D.
	\end{tikzcd}\]
	
	Actually, we need to prove that the definition is well-defined, i.e., to verify that $(\psi\odot\theta)_X\in\Hom_\D(F(X),H(X))$ for all $X\in\Ob(\C)$, it's easy to do so.
\end{defi}


\begin{defi}\label{horizontal composition}
	For functors $F,F':\C_1\to\C_2$ and $G,G':\C_2\to\C_3$, natural transformations $\theta:F\Rightarrow F'$ and $\psi:G\Rightarrow G'$, the element of \emph{horizontal composition} of natural transformations $(\psi\ominus\theta)_X$ is defined as $G'(\theta_X)\circ\psi_{F(X)}=\psi_{F'(X)}\circ G(\theta_X)$. In commutative diagrams forms,
	\[\begin{tikzcd}
			\C_1 \arrow[bend left=50, r, "F", ""' name=F] \arrow[bend right=50, r, "F'"', "" name=F'] &
			\C_2 \arrow[bend left=50, r, "G", ""' name=G] \arrow[bend right=50, r, "G'"', "" name=G'] &
			\arrow[Rightarrow, from=F, to=F', "\theta"] \arrow[Rightarrow, from=G, to=G', "\psi"]\C_3
		\end{tikzcd}\quad{means}\quad\begin{tikzcd}
			\C_1
			\arrow[bend left=50, rr, "GF", ""' name=A]
			\arrow[bend right=50, rr, "G'F'"', "" name=B] & & 
			\arrow[Rightarrow, from=A, to=B, "\psi\ominus\theta"]\C_3,
	\end{tikzcd}\]
	which satisfy
	\begin{equation}\begin{tikzcd}\label{hor}
		GF(X) \arrow[->,>=stealth,r,"G(\theta_X)"] \arrow[->,>=stealth,d,"\psi_{F(X)}"'] & GF'(X) \arrow[->,>=stealth,d,"\psi_{F'(X)}"]\\
		G'F(Y) \arrow[->,>=stealth,r,"G'(\theta_X)"'] & G'F'(Y).
	\end{tikzcd}\end{equation}
	
	Actually, we need to prove that the definition is well-defined, i.e., to verify the commutative diagram and that $(\psi\ominus\theta)_X\in\Hom_{\C_3}(GF(X),G'F'(X))$ for all $X\in\Ob(\C_1)$, it's easy to do so observing \hyperref[ntr]{commutative diagram 1}.
\end{defi}


\begin{thm}
	The vertical and horizontal compositions of natural transformations are natural transformations.	
\end{thm}

\begin{proof}
	For
	\[\begin{tikzcd}
		\C_1
		\arrow[bend left=75, rr, "F", ""' name=F] \arrow[rr, "G" description, "" name=G, ""' name=GG] \arrow[bend right=75, rr, "H"', "" name=H] & &
		\arrow[Rightarrow, from=F, to=G, "\theta"] \arrow[Rightarrow, from=GG, to=H, "\psi"]
		\C_2 \arrow[bend left=75, rr, "F'", ""' name=F'] \arrow[rr, "G'" description, "" name=G', ""' name=GG'] \arrow[bend right=75, rr, "H'"', "" name=H'] & &
		\arrow[Rightarrow, from=F', to=G', "\theta'"] \arrow[Rightarrow, from=GG', to=H', "\psi'"] \C_3
	\end{tikzcd},\]\\
	we need to verify the following commutative diagrams:
		\[(a)\begin{tikzcd}
				F(X) \arrow[->,>=stealth,dd,"F(f)"'] \arrow[->,>=stealth,rr,"(\psi\odot\theta)_X"] && H(X) \arrow[->,>=stealth,dd,"H(f)"]\\ && \\
				F(y) \arrow[->,>=stealth,rr,"(\psi\odot\theta)_X"'] && H(y)
			\end{tikzcd}\quad\text{and (b)}\quad\begin{tikzcd}
				GF(X) \arrow[->,>=stealth,dd,"GF(f)"] \arrow[->,>=stealth,rr,"(\theta'\ominus\theta)_X"] && G'F'(X) \arrow[->,>=stealth,dd,"G'F'(f)"]\\ && \\
				GF(Y) \arrow[->,>=stealth,rr,"(\theta'\ominus\theta)_Y"'] && G'F'(Y)
		\end{tikzcd}.\]
	
	From (a), we have
	\begin{align*}
		& H(f)\circ(\psi\odot\theta)_X\\
		=& H(f)\circ\psi_X\circ\theta_X\tag{Def: vertical composition}\\
		=& (\psi_Y\circ G(f))\circ\theta_X\tag{Property of natural transformation $\psi$}\\
		=& \psi_Y\circ\theta_Y\circ F(f)\tag{Property of natural transformation $\theta$}\\
		=& (\psi\odot\theta)_X\circ F(f),\tag{Def: vertical composition}
	\end{align*}
	thus $(\psi\odot\theta)$ is natural transformation.\\
		
	From (b), we have
	\begin{align*}
		& G'F'(f)\circ(\theta'\ominus\theta)_X\\
		=& G'F'(f)\circ\theta'_{F'(X)}\circ G(\theta_X)\tag{Def: horizontal composition}\\
		=& \theta'_{F'(Y)}\circ GF'(f)\circ G(\theta_X)\tag{Property of natural transformation $\theta'$}\\
		=& \theta'_{F'(Y)}\circ G(F'(f)\circ G(\theta_X))\tag{Property of functor $G$}\\
		=& \theta'_{F'(Y)}\circ G(\theta_Y\circ F(f))\tag{Property of natural transformation $\theta$}\\
		=& \theta'_{F'(Y)}\circ G(\theta_Y)\circ GF(f)\tag{Property of functor $G$}\\
		=& (\theta'\ominus\theta)_Y\circ GF(f)\tag{Def: horizontal composition},
	\end{align*}
	thus $(\psi\ominus\theta)$ is natural transformation.\\
		
	{\bf What's more}, we can prove
		\begin{equation}\label{per}
			(\psi\odot\theta)\ominus(\psi'\odot\theta')=(\psi'\ominus\psi)\odot(\theta'\ominus\theta)
		\end{equation}
	in the below step:
	\begin{align*}
		 & ((\psi'\odot\theta')\ominus(\psi\odot\theta))_X\\
		=& (\psi'\odot\theta')_{H(X)}\circ F'(\psi\odot\theta)_X\tag{Def: horizontal composition}\\
		=& \psi'_{H(X)}\circ\theta_{H(X)}\circ F'(\theta_X)\circ F'(\theta_X)\tag{Def: vertical composition, Property of functor $F'$}\\
		=& \psi'_{H(X)}\circ(G'(\psi_X)\circ \theta'_{G(X)})\circ F'(\theta_X)\tag{\hyperref[hor]{Commutative diagram 2}}\\
		=& (\psi'\ominus\psi)_X\circ(\theta'\ominus\theta)_X\tag{Def: horizontal composition}\\
		=& ((\psi'\ominus\psi)\odot(\theta'\ominus\theta))_X,\tag{Def: vertical composition}
	\end{align*}
	where $X\in\Ob(\C)$.
\end{proof}

\begin{thm}
	Both vertical and horizontal compositions of natural transformations satisfy associative law.
\end{thm}

\begin{proof}
	For vertical composition, observe the following commutative diagram and natural transformations:

	For
	\[\begin{tikzcd}
		\C
		\arrow[rrr, bend left =80, "F", ""' name=F]
		\arrow[rrr, bend left =20, "G" description, "" name=G, ""' name=GG] 
		\arrow[rrr, bend right=20, "H" description, "" name=H, ""' name=HH]
		\arrow[rrr, bend right=80, "K"', "" name=K] & & &
		\D,
		\arrow[Rightarrow, "\theta", from=F, to=G]
		\arrow[Rightarrow, "\psi" , from=GG, to=H]
		\arrow[Rightarrow, "\phi" , from=HH, to=K]
	\end{tikzcd}\]
	it's trivial to prove that $((\phi\odot\psi)\odot\theta)_X=(\phi\odot(\psi\odot\theta))_X$ for all $X\in \Ob(\C)$, thus the vertical composition satisfies associative law.

	For
	\begin{equation}\begin{tikzcd}\label{ass}
		\C_1 \arrow[bend left=50, r, "F", ""' name=F] \arrow[bend right=50, r, "F'"', "" name=F'] &
		\C_2 \arrow[bend left=50, r, "G", ""' name=G] \arrow[bend right=50, r, "G'"', "" name=G'] &	
		\C_3 \arrow[bend left=50, r, "H", ""' name=H] \arrow[bend right=50, r, "H'"', "" name=H'] &
		\arrow[Rightarrow, from=F, to=F', "\theta"]
		\arrow[Rightarrow, from=G, to=G', "\psi"]
		\arrow[Rightarrow, from=H, to=H', "\phi"]
		\C_4,
	\end{tikzcd}\end{equation}
	we have
	\begin{align*}
		 & (\phi\ominus(\psi\ominus\theta))_X\\
		=& \phi_{G'F'(X)}\circ H(\psi\ominus\theta)_X\tag{Def: horizontal composition}\\
		=& \phi_{G'F'(X)}\circ H(\psi_{F'(X)}\circ G(\theta_X))\tag{Ditto}\\
		=& \phi_{G'F'(X)}\circ H(\psi_{F'(X)})\circ HG(\theta_X)\tag{Property of functor $H$}\\
		=& (\phi\ominus\psi)_{F'(X)}\circ HG(\theta_X)\tag{Def: horizontal composition}\\
		=& ((\phi\ominus\psi)\ominus\theta)_X\tag{Ditto},
	\end{align*}
	thus the horizontal composition satisfies associative law.
\end{proof}


\begin{lmm}[$\mathsf{ZF}$]\label{map}
	For any sets $A,B$ and mapping $f:A\to B$, we have
		\[\exists g:B\to A[g\circ f=\id_A]\iff f\ \text{is a injection}\ \iff\forall C\forall h,h':C\to A[f\circ h=f\circ h'\Longrightarrow h=h'],\]
		\[\exists g:B\to A[f\circ g=\id_B]\Longrightarrow f\ \text{is a surjection}\ \iff\forall C\forall h,h':B\to C[h\circ f=h'\circ f\Longrightarrow h=h'],\]
	and $f$ is a surjection $\Longrightarrow \exists g:B\to A[f\circ g=\id_B]$ can be proved in $\mathsf{ZFC}$.
	It's easy to see that $f$ is bijection if and only if it has both left and right inversal mappings, and obviously the two inversal mappings are the same one, which is unique.
\end{lmm}


\begin{defi}\label{inverse}
	Consider $X,Y\in\Ob(\C)$ and $f\in\Hom_\C(X,Y)$. If there exists a morphism $g\in\Hom_\C(Y,X)$ that $f\circ g=\1_Y$, then we say $g$ is the \emph{right inverse} of it; if $g\circ f=\1_X$, then we say $g$ is the \emph{left inverse} of it. If $f$ has both inverses, then we say $f$ is a \emph{isomorphism}, it's easy to verify that the two inverses  are the same one, which is unique as well, so we say $f\iv:=g$ is the \emph{inverse} (\emph{inversal morphism}) of it. If there exists an isomorphism between two objects $X,Y\in\Ob(\C)$, then we say they are isomorphic and record it as $X\mathop{\simeq}\limits^\mathrm{M}Y$.
	
	What's more, it's easy to verify that the composition of isomorphisms is also an isomorphism (see \hyperref[transitivity]{Theorem 3}), so we can find that the collection of \emph{automorphisms} $\mathrm{Aut}_\C(X):=\{f\in\Hom_\C(X,Y)\,|\,f\ \text{is an isomorphism}\}$ is a group $\left<\mathrm{Aut}_\C(X),\circ,\1_X\right>$. Therefore if $f$ has a left inverse, then $f$ is a \emph{injective morphism}, which satisfies the \emph{left cancellation law}: $\forall Z\in\Ob(\C)\forall g,h\in\Hom_\C(Z,X)[f\circ g=f\circ h\iff g=h]$; if $f$ has a right inverse, then $f$ is a \emph{surjective morphism}, which satisfies the \emph{right cancellation law}: $\forall Z\in\Ob(\C)\forall g,h\in\Hom_\C(Y,Z)[g\circ f=h\circ f\iff g=h]$.
	
	Consider a functor $F:\C\to\D$, we can define the inverse of functor in the same way: If there exists a functor $G:\D\to\C$ that $GF=\id_\C$ and $FG=\id_\D$, then we say $F$ is an \emph{isomorphism functor} between $\C$ and $\D$, and $F\iv:=G$ is the unique inverse (\emph{inversal functor}) of $F$. From \hyperref[map]{Lemma 1} we know that $F$ is isomorphic if and only if  $F\upharpoonright_{\Ob(\C)}$ and $F\upharpoonright_{\Mor(\C)}$ are both bijection. If there exists an isomorphism functor between two categories $\C$ and $\D$, we say they are \emph{isomorphic} and record it as $\C\mathop{\simeq}\limits^\mathrm{F}\D$.
	
	(Please see \hyperref[1]{formula 6} first) Consider the following diagram:
	\[\begin{tikzcd}
		\C 
		\arrow[rr, bend left =75, "F", ""' {name=U1, near start}, ""' {name=U2, near end}]
		\arrow[rr, bend right=75, "G"', "" {name=D1, near start}, ""  {name=D2, near end}]
		&&\D.
		\arrow[Rightarrow, "\theta", from=U1, to=D1]
		\arrow[Rightarrow, "\psi"  , from=D2, to=U2]
    \end{tikzcd}\]
	If $\psi\odot\theta=\tid^F$ and $\theta\odot\psi=\tid^G$, then we say $\theta$ is an \emph{isomorphism transformation} between $F$ and $G$, and $\theta\iv:=\psi$ is the unique inverse (\emph{inversal transformation}) of $\theta$, we record it as $\theta:F\mathop{\Rightarrow}\limits^{\sim}G$. It's trivial to prove that $\theta$ is an \emph{isomorphism transformation} if and only if $\theta_X$ is an isomorphism for each $X\in\Ob(\C)$, thus we have $(\theta\iv)_X=(\theta_X)\iv$, we abbreviate it as ``$\theta\iv_X$''. If there exists an isomorphism transformation between two functors $F$ and $G$, we say they are \emph{isomorphic} and record it as $F\mathop{\simeq}\limits^\mathrm{T}G$.
	
	For functors $F:\C\to\D$ and $G:\D\to\C$, if there exists an isomorphic $GF\mathop{\simeq}\limits^\mathrm{T}\id_{\C}$ and $FG\mathop{\simeq}\limits^\mathrm{T}\id_{\D}$, then we say $G$ is the \emph{quasi-inverse} of $F$, and $F$ is a \emph{equivalence} between $\C$ and $\D$. If there exists an equivalence between two categories $\C$ and $\D$, we say they are \emph{equivalent} and record it as $\C\sim\D$.
	
	If there is no confusion in some context, we abbrviate $\mathop{\simeq}\limits^\mathrm{M},\mathop{\simeq}\limits^\mathrm{F},\mathop{\simeq}\limits^\mathrm{T}$ as $\simeq$ and refer to ``isomorphism'', ``isomorphism functor'', ``isomorphism transformation'' as ``isomorphism'' uniformly.
\end{defi}


\begin{lmm}
	Observe \hyperref[ass]{commutative diagram 4}, we have
	\begin{align}
		& G(\tid^F_X)=\tid^G_{F(X)}=\tid^{GF}_X,\label{0}\\
		& \psi\odot\tid^G=\psi=\tid^{G'}\odot\psi,\label{1}\\
		& (\psi\ominus\tid^F)_X=\psi_{F(X)},\label{2}\\
		& (\tid^H\ominus\psi)_Y=H(\psi_Y),\label{3}\\
		& \tid^G\ominus\tid^F=\tid^{GF},\label{4}\\
		& \text{if\ }\psi\text{\ and\ }\phi\text{\ are\ isomorphisms,\ then\ }\phi\ominus\psi\text{\ is\ also,\ and\ }(\phi\ominus\psi)\iv=\phi\iv\ominus\psi\iv,\label{5}\\
		& [F\simeq F'\wedge G\simeq G'\wedge H\simeq H']\Longrightarrow[GF\simeq G'F'\wedge HG\simeq H'G'],\label{6}
	\end{align}
	for all $X\in\Ob(\C_1),Y\in\Ob(\C_2)$.
\end{lmm}

\begin{proof}
	For \hyperref[0]{(5)}, using the definition of identity morphism, properties of morphism and functor we have
		$$\tid^{GF}_X=\1_{GF(X)}=G(\1_{F(X)})=\tid^G_{F(X)}=G(\1_{F(X)})=G(\tid^F_X)$$
	for all $X\in\Ob(\C_1)$.
	
	It's trivial to prove \hyperref[1]{(6)} and using definition of identity transformation and vertical composition.
	
	For \hyperref[2]{(7)}, we have
	\begin{align*}
		 & (\psi\ominus\tid^F)_X\\
		=& \psi_{F(X)}\circ G(\tid^F_X)\tag{Def: horizontal composition}\\
		=& \psi_{F(X)}\circ \tid^G_{F(X)}\tag{\hyperref[0]{Formula 5}}\\
		=& \psi_{F(X)}\tag{\hyperref[1]{Formula 6}}
	\end{align*}
	for all $X\in\Ob(\C_1)$.
	
	For \hyperref[3]{(8)}, we have
	\begin{align*}
		 & (\tid^H\ominus\psi)_Y\\
		=& H(\psi_Y)\circ\tid^H_{G(Y)}\tag{Def: horizontal composition}\\
		=& H(\psi_Y)\circ H(\tid^G_Y)\tag{\hyperref[0]{Formula 5}}\\
		=& H(\psi_Y\circ\tid^G_Y)\tag{Property of functor $H$}\\
		=& H(\psi_Y)\tag{\hyperref[1]{Formula 6}}
	\end{align*}
	for all $Y\in\Ob(\C_2)$.
	
	For \hyperref[4]{(9)}, we have 
	\begin{align*}
		 & (\tid^G\ominus\tid^F)_X\\
		=& G(\tid^F_X)\circ\tid^G_{F(X)}\tag{Def: horizontal composition}\\
		=& \tid^{GF}_X\circ\tid^{GF}_X\tag{\hyperref[0]{Formula 5}}\\
		=& \tid^{GF}_X\tag{\hyperref[1]{Formula 6}}
	\end{align*}
	for all $X\in\Ob(\C_1)$.
	
	For \hyperref[5]{(10)}, we have
	\begin{align*}
		 & (\phi\ominus\psi)\odot(\phi\iv\ominus\psi\iv)\\
		=& (\phi\iv\odot\phi)\ominus(\psi\iv\odot\psi)\tag{\hyperref[per]{Formula 3}}\\
		=& \tid^H\ominus\tid^G\tag{Property of inverse}\\
		=& \tid^{HG}.\tag{\hyperref[4]{Formula 9}}
	\end{align*}
	Similarly, we can prove $(\phi\iv\ominus\psi\iv)\odot(\phi\ominus\psi)=\tid^{H'G'}$.
	
	For \hyperref[6]{(11)}, we suppose that the three natural transformations are all isomorphisms, we claim that $\psi\ominus\theta:GF\equ G'F'$ and $\phi\ominus\psi:HG\equ H'G'$, it's trivial to prove using \hyperref[5]{Formula 10}.
\end{proof}


\begin{thm}\label{transitivity}
	The composition of any morphisms/functors/transformations which are isomorphisms is an isomorphism, and the composition of any equivalences are equivalence. Therefore isomorphic objects, categories, functors and equivalent categories satisfy transitivity.
\end{thm}

\begin{proof}
	Because the compositions of isomorphic morphisms, functors and transformations have some similar algebraic properties, we just need to prove the composition of isomorphic morphisms is isomorphic:\\
	Suppose isomorphisms $f\in\Hom_\C(X,Y),g\in\Hom_\C(Y,Z)$, we claim that $f\iv\circ g\iv$ is the inverse of $g\circ f\in\Hom_\C(X,Z)$:
		$$(g\circ f)\circ(f\iv\circ g\iv)=g\circ(f\circ f\iv)\circ g\iv=g\circ\1_Y
		\circ g\iv=g\circ g\iv=\1_Z,$$
		$$(f\iv\circ g\iv)\circ(g\circ f)=f\iv\circ(g\iv\circ g)\circ f=f\iv\circ\1_Y
		\circ f=f\iv\circ f=\1_X.$$
	
	Observe the following diagram:
	\[\begin{tikzcd}
			\C_1\arrow[loop left, "\id_{\C_1}"]\arrow[bend left=30, r, "F"] &
			\C_2 \arrow[bend left=30, l, "F'"] \arrow[bend left=30, r, "G"] &
			\C_3 \arrow[bend left=30, l, "G'"] \arrow[loop right, "\id_{\C_3}."]
		\end{tikzcd}\]
	We need to prove that if $F$ and $G$ are equivalence then $GF$ is also, we now assume that $F'F\simeq\id_{\C_1}$, $FF'\simeq\id_{\C_2}$, $G'G\simeq\id_{\C_2}$ and $GG'\simeq\id_{\C_3}$. Using \hyperref[6]{formula 11} we have
	$$\id_{\C_2}\simeq G'G\Longrightarrow\id_{\C_1}\simeq F'F=F'\id_{\C_2}F\simeq F'(G'G)F=(F'G')(GF),$$
	$$\id_{\C_2}\simeq F'F\Longrightarrow\id_{\C_3}\simeq GG'=G\id_{\C_2}G'\simeq G(FF')G'=(GF)(F'G').$$
	Using transitivity of isomorphic functors, we have $\id_{\C_1}\simeq(F'G')(GF)$ and $\id_{\C_3}\simeq(GF)(F'G')$. Thus $GF$ is equivalence.
\end{proof}


\begin{cor}
	If functors $G,G'$ are quasi-inverses of equivalence $F:\C\to\D$, then $G\simeq G'$.
\end{cor}

\begin{proof}
	Using \hyperref[6]{formula 11}, we have
	\[G'F\simeq\id_\C\wedge FG\simeq\id_\D\Longrightarrow G'=G'\id_\D\simeq G'(FG)=(G'F)G\simeq\id_\C G=G,\]
	thus we have $G'\simeq G$ using transitivity of isomorphic functors.
\end{proof}


\begin{defi}\label{different functors}
	For functor $F:\C\to\D$,
	\begin{itemize}
		\item $F$ is \emph{essentially surjective} if $\forall Y\in\Ob(\D)\exists X\in\Ob(\C)[F(X)\mathop{\simeq}\limits^\mathrm{M}F(Y)]$;
		\item $F$ is \emph{faithful} if for all $X,Y\in\Ob(\C)$, $F\upharpoonright_{\Hom_\C(X,Y)}\to\Hom_\C(F(X),F(Y))$ is injective;
		\item $F$ is \emph{full} if for all $X,Y\in\Ob(\C)$, $F\upharpoonright_{\Hom_\C(X,Y)}\to\Hom_\C(F(X),F(Y))$ is surjective.
	\end{itemize}
\end{defi}


\begin{defi}\label{skeleton}
	A full subcategory $\C'$ of category $\C$ is a \emph{skeleton} of $\C$ if $\forall X\in\Ob(\C)\exists! Y\in\Ob(\C')[X\mathop{\simeq}\limits^\mathrm{M}Y]$. If $\C$ is the skeleton of itself, then we say it's a \emph{skeletal} category.
\end{defi}

\end{document}
